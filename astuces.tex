\documentclass[a4paper,10pt]{article}
\usepackage[utf8]{inputenc}
\usepackage{amsfonts}
\usepackage{amsmath}

%opening
\title{Astuces pour l'analyse}
\author{}

\begin{document}

\maketitle

\begin{abstract}

\end{abstract}
\section{coefficients binomiaux}
\[\textstyle{n \choose k} = C^{k}_{n} = \frac{n!}{(n - k)!k!}\]
\[\textstyle{n \choose k} + \textstyle{n \choose k + 1} = \textstyle{n + 1 \choose k + 1}\]
\[(a + b)^{n} = \sum_{k = 0}^{n} \textstyle{n \choose k} a^{k}b^{n - k}\]
\[\sum_{k = 0}^{n} C^{k}_{n} = 2^{n}\]

\section{Linéarisation}
Avant de dériver des fonctions trigonométriques toujours penser à linéariser.
\begin{align*}
sin^{3}(x) &= (\frac{e^{ix} - e^{-ix}}{2})^{3}\\
&= \frac{e^{3ix} - 3e^{ix} + 3e^{-ix} - e^{-3ix}}{8}\\
&= \frac{(e^{3ix} - e^{-3ix}) -(3e^{ix} - 3e^{-ix})}{8}\\
&= \frac{2sin(3x) - 2*3sin(x)}{8}\\
&= \frac{sin(3x)}{4} - \frac{3sin(x)}{4}
\end{align*}
\section{contraposé}
\[(a + b)(a - b) = a^{2} - b^{2}\]
\[a - b = \frac{a^{2} - b^{2}}{a + b}\]
\section{}
\[\frac{1}{x(1 - x)} = \frac{1}{x} + \frac{1}{1 - x}\]
\[|\sqrt{a} - \sqrt{b}| <= \sqrt{|a - b|}\]
\appendix
\section{Preuves}
\subsection{coefficients binomiaux}
\begin{align*}
 \textstyle{n \choose k} + \textstyle{n \choose k + 1} &= \frac{n!}{(n - k)!k!} + \frac{n!}{(n - (k + 1)!(k + 1)!}\\
 &= \frac{n!(n - k + k + 1)}{(n - k)!(k + 1)!}\\
 &= \frac{n!(n + 1)}{((n + 1) - (k + 1))!(k + 1)!}\\
 &= \textstyle{n + 1 \choose k + 1}
\end{align*}
\[(1 + 1)^{n} = \sum_{k = 0}^{n} \textstyle{n \choose k} 1^{k}1^{n - k} = \sum_{k = 0}^{n} \textstyle{n \choose k}\]


\end{document}
