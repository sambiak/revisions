\documentclass[a4paper,10pt]{article}
\usepackage[utf8]{inputenc}
\usepackage{amsfonts}
\usepackage{amsmath}
\usepackage{amssymb}

%opening
\title{Astuces pour l'analyse}
\author{}

\begin{document}

\maketitle

\begin{abstract}

\end{abstract}
\section{coefficients binomiaux}
\[\textstyle{n \choose k} = C^{k}_{n} = \frac{n!}{(n - k)!k!}\]
\[\textstyle{n \choose k} + \textstyle{n \choose k + 1} = \textstyle{n + 1 \choose k + 1}\]
\[(a + b)^{n} = \sum_{k = 0}^{n} \textstyle{n \choose k} a^{k}b^{n - k}\]
\[\sum_{k = 0}^{n} C^{k}_{n} = 2^{n}\]

\section{Linéarisation}
Avant de dériver des fonctions trigonométriques toujours penser à linéariser.
\begin{align*}
sin^{3}(x) &= (\frac{e^{ix} - e^{-ix}}{2})^{3}\\
&= \frac{e^{3ix} - 3e^{ix} + 3e^{-ix} - e^{-3ix}}{8}\\
&= \frac{(e^{3ix} - e^{-3ix}) -(3e^{ix} - 3e^{-ix})}{8}\\
&= \frac{2sin(3x) - 2*3sin(x)}{8}\\
&= \frac{sin(3x)}{4} - \frac{3sin(x)}{4}
\end{align*}
\section{contraposé}
\[(a + b)(a - b) = a^{2} - b^{2}\]
\[a - b = \frac{a^{2} - b^{2}}{a + b}\]
\section{Divers}
\[\frac{1}{x(1 - x)} = \frac{1}{x} + \frac{1}{1 - x}\]
\[|\sqrt{a} - \sqrt{b}| <= \sqrt{|a - b|}\]
\[|sin(x)| <= |x|\]
\section{équivalence}
\[cos(x) - 1 = \frac{-x^2}{2}\]
\[sin(x) = x\]
\section[Aussi nommé Théorème de Weierstrass.]{Théorème des valeurs extrêmes}



Soit $[a; b]$ un intervalle fermé borné de R. Soit $f \colon [a;b] \longrightarrow \mathbb{R}$ une fonction continue sur cette intervalle. Alors
$f([a;b])$ est une partie bornée de R et il existe $x_{1} \in [a;b]$ et $x_{2} \in [a;b]$ tels que $f(x_{1}) = sup f([a;b])$ et $f(x_{2}) = inf f([a;b])$





On raisonne par l'absurde soit  $[a; b]$ un intervalle fermé borné de R. Soit $f \colon [a;b] \longrightarrow \mathbb{R}$ une fonction continue sur cette intervalle.
On suppose que f n'est pas bornée pour simplifier la démonstration on supposera qu'elle n'est pas bornée par le haut. 
Il existe alors $(x_{n})_{n \in \mathbb{N}}$une suite de points de $[a; b]$ tels que $f(x_{n}) > n$. 
On a \[\lim_{n \to \infty} f(x_{n}) = +\infty\]
Donc la suite $(f(x_{n}))_{n \in \mathbb{N}}$ diverge.
Or comme $(x_{n})_{n \in \mathbb{N}}$ est bornée d'aprés le Théorème de bolzano-Weierstrass il existe une
extractrice $\phi \colon \mathbb{N} \longrightarrow \mathbb{N}$ tel que $(x_{\phi(n)})_{n \in \mathbb{N}}$ converge vers une valeur $c \in [a; b]$ car 
$[a;b]$ est bornée.

Or comme $f$ est continue \[\lim_{n \to \infty} f(x_{\phi(n)}) = f(\lim_{n \to \infty} x_{\phi(n)}) = f(c)\]
La suite $(f(x_{\phi(n)}))_{n \in \mathbb{N}}$ converge donc vers $f(c)$. Ceci est absurde car la suite $(f(x_{\phi(n)}))_{n \in \mathbb{N}}$ est
une sous suite de $(f(x_{n}))_{n \in \mathbb{N}}$ qui diverge.


On en déduit que $f([a; b])$ est une partie bornée de $\mathbb{R}$


On cherche maintenant a dèmontrer qu'il existe $c \in [a;b]$ tel que $f(c) = sup f([a; b])$
D'aprés la définition du sup il éxiste une suite de points de $[a; b]$. $(x_{n})_{n \in \mathbb{N}}$ tel que $sup f([a; b]) > f(x_{n}) >sup f([a; b]) -\frac{1}{n}$
D'aprés le théoreme des gendarmes :
\[\lim_{n \to \infty} f(x_{n}) = sup f([a; b])\]
Or comme $(x_{n})_{n \in \mathbb{N}}$ est bornée d'aprés le Théorème de bolzano-Weierstrass il existe une
extractrice $\phi \colon \mathbb{N} \longrightarrow \mathbb{N}$ tel que $(x_{\phi(n)})_{n \in \mathbb{N}}$ converge vers une valeur $c \in [a; b]$ car 
$[a;b]$ est bornée. Or comme f continue.
\[f(c) = f(\lim_{n \to \infty} x_{\phi(n)}) = \lim_{n \to \infty} f(x_{\phi(n)}) = sup f([a; b])\]
\section{Théoreme de Rolle}
Soit $f$ une fonction continue sur $[a; b]$ et dérivable sur $]a;b[$ tel que $f(a) = f(b)$ alors il
existe $c \in ]a;b[$ tel que $f'(c) = 0$

\subsection{Démo}
Deux cas soit $f$ constante soit pas.


Si $f$ constante $f'(x) = 0$ $ \forall x \in ]a;b[$


Sinon comme $f$ n'est pas constante on à $sup f([a; b]) > f(a)$ ou $min f([a; b]) < f(a)$
supposons par exemple que $sup f([a; b]) > f(a)$ alors d'aprés Weierstrass il éxiste $c \in [a; b]$
tel que $f(c) = sup f([a; b])$ on a $f(c) > f(a) = f(b)$ donc $c \in ]a;b[$
\[\lim_{h \to 0^{+}} \frac{f(c + h) - f(c)}{h} \leq 0\]
\[\lim_{h \to 0^{-}} \frac{f(c + h) - f(c)}{h} \geq 0\]
Comme $f$ continue \[\lim_{h \to 0^{+}} \frac{f(c + h) - f(c)}{h} = \lim_{h \to 0^{-}} \frac{f(c + h) - f(c)}{h}\]
donc $f'(c) = 0$
\section{Théoreme des accroissements finis}
Soit $f$ une fonction continue sur $[a; b]$ et dérivable sur $]a;b[$ alors il éxiste
$c \in ]a;b[$ tel que $f'(c) = \frac{f(b) - f(a)}{b - a}$


On cherche une fonction $u$ tel $u(a) = u(b)$ pour pouvoir appliquer rolle dessus. Pour ce faire on pose $u(x) = f(x) + \lambda (x - a)$.
On a $u(a) = f(a)$ on cherche a faire que $u(b) = f(a)$.


On veut 
\begin{align*}
f(a) &= f(b) + \lambda(b - a)\\
f(a) - f(b) &= \lambda(b - a)\\
\frac{f(a) - f(b)}{b - a} &= \lambda
\end{align*}
Il suffit de poser $\lambda = \frac{f(a) - f(b)}{b - a}$
La fonction $u$ est une somme de fonction continue sur $[a;b]$ et dérivables sur $]a; b[$
D'aprés rolle il existe $c \in ]a;b[$ tel que $u'(c) = 0$ or
\[0 = u'(c) = f'(c) + \frac{f(a) - f(b)}{b - a}\]
donc \[f'(c) =  \frac{f(b) - f(a)}{b - a}\]

\section{théoreme de cauchy}
Soinet $g$ et $f$ fonctions continues sur $[a; b]$ et dérivables sur $]a;b[$ tel que $g(b) - g(a) \neq 0$
alors il existe $c \in ]a;b[$ tel que $f'(c) = g'(c)\frac{f(b) - f(a)}{g(b) - g(a)}$


\section{démo}

On cherche une fonction $u$ tel $u(a) = u(b)$ pour pouvoir appliquer rolle dessus. Pour ce faire on pose
$u(x) = f(x) + \lambda (g(x) - g(a))$
on a $u(a) = f(a)$



On veut
\begin{align*}
f(a) &= f(b) + \lambda(g(b) - g(a))\\
f(a) - f(b) &= \lambda(g(b) - g(a))\\
\frac{f(a) - f(b)}{g(b) - g(a)} &= \lambda
\end{align*}

$u$ comme somme de fonctions dérivables est dérivable sur $]a;b[$ et comme somme de fonctions continues est continue sur $[a;b]$
d'aprés rolle il existe $c \in ]a;b[$ tel $u'(c) = 0$ or
\[0 = u'(c) = f'(c) + g'(c)\frac{f(a) - f(b)}{g(b) - g(a)}\]
donc \[f'(c) = g'(c)\frac{f(b) - f(a)}{b - a}\]







\section{Théorème de heine}
Toute fonction continue sur $[a;b]$ intervalle de $\mathbb{R}$ est uniformément continue sur ce segment.
\section{Théorème de leibinz}
Soit n un entier positif. Le produit de deux fonctions dérivables jusqu'a l'ordre n est dérivable jusqu'a l'ordre n et
\[(fg)^{n} = \sum_{k = 0}^{n}C^{k}_{n}f^{(k)}g^{(n-k)}\]
\subsection{dèmo}
On démontre par réccurence.
\subsubsection[cela marche pas]{Intitialisation}
\[(fg)^{0} = \sum_{k = 0}^{0}C^{0}_{0}f^{(0)}g^{(0-0)} = fg\]
\subsubsection{hérédité}
On a \[(fg)^{n} = \sum_{k = 0}^{n}C^{k}_{n}f^{(k)}g^{(n-k)}\] supposons que f et g soient dérivables a l'ordre n + 1.
$(fg)^{n}$ est donc une somme de produits de fonctions dérivables. $(fg)^{n}$ est donc dérivable et ca dérivée vaut.
\begin{align*}
(fg)^{n + 1} &= \sum_{k = 0}^{n}C^{k}_{n}(f^{(k + 1)}g^{(n-k)} + f^{(k)}g^{(n + 1 -k)})\\
&= \sum_{k = 0}^{n}C^{k}_{n}f^{(k + 1)}g^{(n-k)} + C^{k}_{n}f^{(k)}g^{(n + 1 -k)})\\
&= \sum_{k = 0}^{n}C^{k}_{n}f^{(k + 1)}g^{(n-k)} + \sum_{k = 0}^{n}C^{k}_{n}f^{(k)}g^{(n + 1 -k)}\\
&= \sum_{j = 1}^{n + 1}C^{j - 1}_{n}f^{(j)}g^{(n + 1 -j)} + \sum_{k = 0}^{n}C^{k}_{n}f^{(k)}g^{(n + 1 -k)}\\
&=C^{n}_{n}f^{n + 1}g^{0} +  \sum_{j = 1}^{n}C^{j - 1}_{n}f^{(j)}g^{(n + 1 -j)} + \sum_{k = 1}^{n}C^{k}_{n}f^{(k)}g^{(n + 1 -k)} + C^{0}_{n}f^{(0)}g^{(n + 1)}\\
&=C^{n + 1}_{n + 1}f^{n + 1}g^{0} +  \sum_{j = 1}^{n}(C^{j - 1}_{n} + C^{j}_{n})f^{(j)}g^{(n + 1 -j)} + C^{0}_{n + 1}f^{(0)}g^{(n + 1)}\\
&=C^{n + 1}_{n + 1}f^{n + 1}g^{0} +  \sum_{j = 1}^{n}(C^{j}_{n + 1})f^{(j)}g^{(n + 1 -j)} + C^{0}_{n + 1}f^{(0)}g^{(n + 1)}\\
&= \sum_{k = 0}^{n + 1}C^{k}_{n + 1}f^{(k)}g^{(n + 1-k)}
\end{align*}
\appendix
\section{Demonstrations}
\subsection{coefficients binomiaux}
\begin{align*}
 \textstyle{n \choose k} + \textstyle{n \choose k + 1} &= \frac{n!}{(n - k)!k!} + \frac{n!}{(n - (k + 1)!(k + 1)!}\\
 &= \frac{n!(n - k + k + 1)}{(n - k)!(k + 1)!}\\
 &= \frac{n!(n + 1)}{((n + 1) - (k + 1))!(k + 1)!}\\
 &= \textstyle{n + 1 \choose k + 1}
\end{align*}
\[(1 + 1)^{n} = \sum_{k = 0}^{n} \textstyle{n \choose k} 1^{k}1^{n - k} = \sum_{k = 0}^{n} \textstyle{n \choose k}\]
\subsection{divers}
Si $a >= b$
\begin{align*}
 0 &<= 2\sqrt{b}(\sqrt{a} - \sqrt{b})\\
 0 &<= 2\sqrt{a}\sqrt{b} -2b\\
 a - 2\sqrt{a}\sqrt{b} + b &<=  a - b\\
 |\sqrt{a} - \sqrt{b}|^{2} &<= \sqrt{|a - b|}^2\\
 |\sqrt{a} - \sqrt{b}| &<= \sqrt{|a - b|}
\end{align*}
même idée si $b <= a$


Pour $\forall x \in \mathbb{R}^{+}$
\begin{align*}
 cos(x) &<= 1\\
 \int_{0}^{x} cos(x) dx&<= \int_{0}^{x} dx\\
 sin(x) &<= x
\end{align*}
Comme $x \longrightarrow x$ et $sin(x)$ impaires $\forall x \in \mathbb{R}$ 
\[|sin(x)| <= |x|\]




\end{document}
